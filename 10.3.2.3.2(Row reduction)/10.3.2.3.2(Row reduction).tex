\documentclass[12pt]{article}
\usepackage{amsmath}
\newcommand{\myvec}[1]{\ensuremath{\begin{pmatrix}#1\end{pmatrix}}}
\newcommand{\mydet}[1]{\ensuremath{\begin{vmatrix}#1\end{vmatrix}}}
\newcommand{\solution}{\noindent \textbf{Solution: }}
\providecommand{\brak}[1]{\ensuremath{\left(#1\right)}}
\providecommand{\norm}[1]{\left\lVert#1\right\rVert}
\let\vec\mathbf

\title{Linear Equations in Two Variables}
\author{potnurudeekshitha (potnurudeekshitha@sriprakashschools.com)}

\begin{document}
\maketitle
\section*{10$^{th}$ Maths - Chapter 3}
This is Problem-2 from Exercise 3.2
\begin{enumerate}
\item  Which of the following pairs of linear equations are consistent/inconsistent? If consistent, obtain the solution graphically:

(i) x + y = 5, 2x + 2y = 10 \\\\
\end{enumerate}
\solution \\
Given Data:\\
            9x+3y=-12\\ 
            18x+6y=-24\\

This can also be written as:
\begin{align}
\myvec{9&3&-12\\18&6&-24}
\end{align}
now,Making $R_2 \xrightarrow\ 2R_1 - R_2$\\ 
we get,
\begin{align}
\myvec{9&3&-12\\0&0&0}
\end{align}
Since, we are getting zero in $R_2$\\
It is a dependent equation.
\end{document}