\documentclass[12pt]{article}
\usepackage{amsmath}
\newcommand{\myvec}[1]{\ensuremath{\begin{pmatrix}#1\end{pmatrix}}}
\newcommand{\mydet}[1]{\ensuremath{\begin{vmatrix}#1\end{vmatrix}}}
\newcommand{\solution}{\noindent \textbf{Solution: }}
\providecommand{\brak}[1]{\ensuremath{\left(#1\right)}}
\providecommand{\norm}[1]{\left\lVert#1\right\rVert}
\let\vec\mathbf

\title{Linear equations in two variables}
\author{Paidisetty Rithik(paidisettyrithik@sriprakashschools.com)}
\begin{document}
\maketitle
\section*{10$^{th}$ Maths - Chapter 3}
This is Problem-3.5 from Exercise 3.2
\begin{enumerate}
\item On comparing the ratios $\frac{a_1}{a_2}$,$\frac{b_1}{b_2}$and$\frac{c_1}{c_2}$,find out whether the following pairs of linear equations are consistent,or inconsistent\\
\begin{align}
    \frac{4}{3}x+2y=8\\
       2x+3y=12
\end{align}

\solution \\
Matrix form of the equations:
$\myvec{\frac{4}{3} & 2 & 8\\2 & 3 & 12}$\\
$R_1=\myvec{\frac{4}{3} & 2 & 8},R_2=\myvec{2 & 3 & 12}$\\
$R_1\rightarrow3R_1$,we get:
\begin{align}
\myvec{4 & 6 & 24\\2 & 3 & 12}
\end{align}
$R_1\rightarrow\frac{R_1}{2}$,we get:
\begin{align}
\myvec{2 & 3 & 12\\2 & 3 & 12}
\end{align}
$R_2\rightarrow$$R_2-R_1$
\begin{align}
\myvec{2 & 3 & 12\\0 & 0 & 0}
\end{align}
As $R_2$ is equal to$\myvec{0 & 0 & 0}$,Therefore the two equations have infinitely many solutions

\end{enumerate}



\end{document}