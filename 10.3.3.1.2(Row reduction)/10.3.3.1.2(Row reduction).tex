\documentclass[12pt]{article}
\usepackage{amsmath}
\newcommand{\myvec}[1]{\ensuremath{\begin{pmatrix}#1\end{pmatrix}}}
\newcommand{\mydet}[1]{\ensuremath{\begin{vmatrix}#1\end{vmatrix}}}
\newcommand{\solution}{\noindent \textbf{Solution: }}
\providecommand{\brak}[1]{\ensuremath{\left(#1\right)}}
\providecommand{\norm}[1]{\left\lVert#1\right\rVert}
\let\vec\mathbf
\title{Linear Equations in Two Variables}
\author{karthik pyla(karthik.pyla@sriprakashschools.com)}
\begin{document}
\maketitle
\section*{10$^{th}$ Maths - Chapter 3}
This is Problem-(1)ii from Exercise 3.3
\begin{enumerate}
\item On comparing the ratios $\frac{a_1}{a_2}$ , $\frac{b_1}{b_2}$ ,$\frac{c_1}{c_2}$, find out whether the lines representing the following pairs of linear equations intersect at a point, are parallel or coincident:\\
x-y=3\\ 
2x-3y =36\\
\end{enumerate}

\solution \\
Matrix form of the equations:
$\myvec{1 & -1 & 3\\2 & -3 & 36}$\\
$R_1=\myvec{1 & -1 & 3},R_2=\myvec{2 & 3 & 36}$\\
$R_2\rightarrow R_2 - 2R_1$,we get:
\begin{align}
\myvec{1 & -1 & 3\\0 & -1 & 30}
\end{align}
$R_2\rightarrow \frac{R_2}{-1}$ ,we get:
\begin{align}
\myvec{1 & -1 & 3\\0 & 1 & -30}
\end{align}
$R_1\rightarrow R_1 + R_2 $

\begin{align}
\myvec{1 & 0 & -27\\0 & 1 & -30}
\end{align}
Therefore,x = -27 , y = -30
\end{document}
