\documentclass[12pt]{article}
\usepackage{amsmath}
\newcommand{\myvec}[1]{\ensuremath{\begin{pmatrix}#1\end{pmatrix}}}
\newcommand{\mydet}[1]{\ensuremath{\begin{vmatrix}#1\end{vmatrix}}}
\newcommand{\solution}{\noindent \textbf{Solution: }}
\providecommand{\brak}[1]{\ensuremath{\left(#1\right)}}
\providecommand{\norm}[1]{\left\lVert#1\right\rVert}
\let\vec\mathbf
\title{Linear Equations in Two Variables}
\author{N.Charan(charan.n@sriprakashschools.com)}
\begin{document}
\maketitle
\section*{10$^{th}$ Maths - Chapter 3}
This is Problem-(1)i from Exercise 3.3
\begin{enumerate}
\item On comparing the ratios $\frac{a_1}{a_2}$ , $\frac{b_1}{b_2}$ ,$\frac{c_1}{c_2}$, find out whether the lines representing the following pairs of linear equations intersect at a point, are parallel or coincident:\\
x+y=14\\ 
x-y=4\\
\end{enumerate}

\solution \\
Matrix form of the equations:
$\myvec{1 & 1 & 14\\1 & -1 & 4}$\\
$R_1=\myvec{1 & 1 & 14},R_2=\myvec{1 & -1 & 4}$\\
$R_1\rightarrow R_1 + R_2$,we get:
\begin{align}
\myvec{2 & 0 & 18\\1 & -1 & 4}
\end{align}
$R_2\rightarrow 2R_2 - R_1$ ,we get:
\begin{align}
\myvec{2 & 0 & 18\\0 & -2 & -10}
\end{align}
$R_1\rightarrow \frac{R_1}{2}\\
R_2\rightarrow \frac{R_2}{-2}

\begin{align}
\myvec{1 & 0 & 9\\0 & 1 & 5}
\end{align}

Therefore,x = 9 , y = 5

\end{enumerate}
\end{document}
